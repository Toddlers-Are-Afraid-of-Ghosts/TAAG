\documentclass[12pt,a4paper]{article}
\usepackage[margin = 3cm]{geometry}
\usepackage[utf8x]{inputenc}
\usepackage[francais]{babel}
\usepackage{graphicx}
\usepackage[T1]{fontenc}
\usepackage{dirtytalk}
\usepackage[table,xcdraw]{xcolor}
\usepackage{fancyhdr}
\usepackage{caption}
\usepackage{lmodern}
\usepackage{changepage}
\usepackage{textgreek}
\usepackage{soul}
\usepackage{hyperref}
\usepackage{xspace}
\usepackage{amsmath}
\usepackage{gensymb}
\documentclass[10pt,a4paper]{article}
\usepackage[utf8]{inputenc}
\usepackage[T1]{fontenc}
\usepackage[french]{babel}
\usepackage{array,multirow,makecell}
\documentclass[10pt,a4paper]{article}
\usepackage[utf8]{inputenc}
\usepackage[T1]{fontenc}
\usepackage[french]{babel}
\documentclass[10pt,a4paper]{article}
\usepackage[utf8]{inputenc}
\usepackage[T1]{fontenc}
\usepackage[french]{babel}
\usepackage{array,multirow,makecell}


%-----------------------------------
%---------------Code----------------
%-----------------------------------

\begin{document}

\begin{center}
    {\huge\textbf{
        Cahier des charges
    }}
    \vspace*{\fill}
    \\
    \includegraphics[width = 15cm]{Image/final-logo.png}
     \vspace*{\fill}
    \\
    par
    \\
    \vspace{.2cm}
    {\huge\textbf{
    TAAG TEAM}}
    \\
    \vspace{1 cm}
    composé de :
    \vspace{0.2cm}
    {\textbf{
        \\
        Thomas Solatges\\
        Arnaud Corcione\\
        Axelle Destombes\\
        Gwennan Jano\\
    }}
\end{center}
\newpage
\vspace*{1mm}
\tableofcontents

%--------En-tète------------
\pagestyle{fancy}
\lhead{Toddlers Are Afraid of Ghosts}
\rhead{TAAG-Team}
%---------------------------

\newpage
\vspace*{1mm}
\section{Introduction}
\paragraph{}
    Les fantômes ou les monstres en général existent-ils vraiment ? Ou sont-ils seulement des créatures utilisées par les adultes afin de faire peur aux enfants ? Bien que les membres de la TAAG TEAM n’en sont plus vraiment, ils persistent pourtant à craindre en quelque sorte ces mystérieuses entités. 
    \\
    \vspace*{0.2cm}
    \\
    Toddlers Are Afraid of Ghost s’inspire donc de nos peurs enfantines et nous permet de combattre ces monstres. 
    \\
    \vspace*{0.2cm}
    \\
    En vous glissant dans les personnages des développeurs du jeu, vous pourrez explorer différents univers, affronter vos peurs et vous amuser. 
\\
\vspace*{.1cm}
\section{Présentation des membres}
\subsection{Thomas "VALI" Solatges}
\paragraph{}
Aussi connu sous le nom de « VALI » (non, ça n'a rien a voir avec Valentin, il faut arrêter de me poser cette question, merci), je suis passionné par les jeux vidéo depuis ma première DS et la découverte de Pokémon. Depuis, je ne cesse d’être émerveillé par les technologies développées chaque année, et qui ne limitent les créateurs qu’à leur imagination.
\\
\\
Je suis de nature assez tranquille, bien que certaines activités brisent cette tendance. J’ai un esprit compétitif, et je prends très à cœur le défi, c’est d’ailleurs ma principale source de sel (et ne vous en faites pas, c’est du bon). Je suis aussi un gros fan de musique électronique, dont je me limite dans l’écoute, car j’ai un minimum de respect pour ce genre. Armé de Buck la Belette et de beaucoup de patience (je me suis quand même infligé 300 heures de The Binding of Isaac), je compte bien faire de ce projet une réussite.
\\
\newpage
\vspace*{.1cm}
\subsection{Arnaud "Arnoloh" Corcione}
\paragraph{}
De mon appellation « Arnoloh » ou Arno, pour les intimes, je suis un grand fan de jeux vidéo ainsi que de la programmation, d’où ma présence à l’EPITA. Mon plus grand trait de personnalité est certainement ma folie constante. J’ai le rire facile, et pour moi, rien ne vaut plus qu'une bonne blague. La joie et la bonne humeur sont des choses de très importantes.
\\
\\
Les jeux vidéo m’ont bercé depuis que je suis minot. Alors, en réaliser un est d’évident pour moi. Surtout que j’ai eu la chance de faire le jeu « Pong » en JavaScript. Depuis, mon goût pour la programmation et les jeux vidéo n'a cessé d’augmenter chaque jour.
\\
\\
La musique est aussi quelque chose d'essentiel dans mon quotidien. C'est pourquoi, j'adore faire du bruit ou de jouer du ukulélé, un instrument que mes camarades n'auront pas fini d'entendre.
\\
De plus, niveau programmation, je me suis aussi penché sur le coding d’un bot discord, qui n’est certes plus d’actualité mais il s’appelait Robert !
C’est pourquoi Attilio, mon Rubber Ducky, et moi, sommes plus que motivé pour ce projet.
\\
 \vspace*{.2cm}
    \\
   \hspace*{6.5cm} \includegraphics[width = 2cm]{Image/attilio.png}
    \begin{adjustwidth}{20pt}{20pt}
    \centering{\textit{
    «Coin Coin»}
    \\[5pt]
    \rightline{{\rm --- Attilio}}}
    \end{adjustwidth}
    \\
\newpage
\vspace*{.1cm}
\subsection{Axelle "Lunarya" Destombes}
\paragraph{}
Bonjour ! Je suis Axelle, Lunarya ou bien Luna (rien d'original, seulement mon deuxième prénom ;) ). Passionnée de jeux vidéos, de lecture et de musique depuis mon enfance, je suis extrêmement enthousiaste à l'idée de participer à ce projet. Depuis quelques années, la programmation occupe une quantité importante de mon temps. 
\\
Il m'est plusieurs fois arrivé de créer des outils qui me sont par exemple utiles pour optimiser mon farm dans certains des jeux auxquels je joue. Certes, pour toute autre personne, ces outils seraient probablement inutiles, mais j'aime bien programmer, d'autant plus lorsque cela inclut un autre de mes centres d'intérêt.
\\
\\
De nature plutôt calme et posée, il semblerait que je sois toute désignée pour devenir la "maman" de ce groupe. J'ai hâte de passer de merveilleux moments avec mes trois camarades. Les prochains mois vont certainement être remplis de bonne humeur, de rires et de fatigue. Je n'ai cependant aucun doute sur notre capacité à réaliser ce projet qui nous tient tous à cœur. 
\\


\subsection{Gwennan "Liberty" Jarno}
\paragraph{}
Connue sous le pseudonyme de « Liberty » pour les uns et « Gwen » pour les autres, ces petits noms risquent de changer incessamment sous peu du fait de mon envie d'avoir moi aussi un nom stylé.
\\
Complète débutante dans tout ce qui touche aux jeux vidéos, j'adore explorer et créer de nouvelles choses. Je serais visiblement la prof de français de ce groupe ainsi que la dessinatrice. 
\\
Au fur et à mesure que j'apprends à connaître mes compatriotes de projet, je dois avouer que j'ai de plus en plus de doutes sur la "santé mentale" de notre futur projet.
\\
Je serais sûrement aussi le petit Schroumpf du groupe, qui s'amusera à bien embêter ses camarades aussi peu sérieux qu'elle.
\\
\\
Mais, en dépit de cette présentation quelque peu non sérieuse, je peux vous assurer de ma motivation et de mon envie de réaliser et faire aboutir ce projet.
\\
 \vspace*{.2cm}
    \\
   \hspace*{5cm} \includegraphics[width = 5cm]{Image/stchtroumpf.png}
\newpage
\vspace*{1mm}
\section{Origine du projet}
La création d’un jeu vidéo fut une envie commune dans le groupe. L’idée d’un jeu semblable à The Binding of Isaac est ensuite apparue comme une évidence au sein du groupe.
\vspace*{0.2cm}
\\
Toddlers Are Afraid est donc un jeu de type roguelike. Le joueur affronte des monstres et parcourt des niveaux générés de façon procédurale. À chaque fois qu’il perd une partie, le joueur doit recommencer depuis le début dans une nouvelle, sans aucun inventaire. 
\vspace*{.2cm}

\section{Objet de l'étude}
\paragraph{}
\normalsize
{
D'un point de vue collectif, ce projet a pour but de nous faire travailler en groupe, ainsi que de nous faire développer un esprit d’équipe et d’entraide face aux problèmes rencontrés. De plus, la création d’un jeu nous permettra d’acquérir à la fois de nouvelles connaissances en développement mais aussi en organisation. 
\vspace*{0.2cm}
\\
D'un point de vue individuel, chacun souhaite découvrir et approfondir les tâches et parties du jeu qui lui ont été confiées. 
\vspace*{0.2cm}
\\
Notre objectif commun est de répondre aux attentes et contraintes imposées et d’avoir une bonne note. }
\\

\section{État de l'art}
\subsection{\textbf{Premiers logiciels/jeux de ce type}}
\\
\vspace*{.1cm}
\\
\normalsize{\subsubsection{Beneath Apple Manor en 1978}}
\\
Beneath Apple Manor est un roguelike sorti en 1978 et publié par The Software factor pour l’Apple II, . C’est le premier jeu à générer procéduralement ses niveaux. Bien que le jeu soit catégorisé en tant que roguelike, il est sorti avant Rogue, jeu considéré comme le pilier du genre. Beneath Apple Manor se joue sur un écran qui se révèle au fur et à mesure que le joueur se déplace. Les actions possibles à faire sur chaque case sont affichées sous forme de texte. Cela permet de rendre le jeu clair et de faire comprendre l’histoire au joueur à une époque où les graphismes sont encore très simples. 
\\
\newpage
\vspace*{.1cm}
\normalsize{\subsubsection{Rogue en 1980}}
\\
Rogue est un jeu sorti en 1980 pour des systèmes sous Unix, et distribué gratuitement. Cela a largement contribué à son succès et à son titre de pilier du genre, auquel il a laissé son nom. Les graphismes du jeu sont en ASCII,ce qui a permis une plus grande liberté (le jeu prend moins d'espace sur le disque, à une époque où cet espace est très restreint). C’est ce que l'on appelle aujourd’hui un jeu de terminal. Rogue intègre une mort permanente, ce qui fait de ce jeu un die and retry, c’est-à-dire que l'on acquiert des connaissances de jeu et de l’expérience en mourant. Cette mécanique permet aussi de réduire le temps des parties, tout en augmentant la durée de vie du jeu. 
\\

\vspace*{1mm}
\subsection{\textbf {Principaux jeux de ce type existant}}
\\
\normalsize{\subsubsection{The Binding of Isaac}}
The Binding of Isaac est un jeu paru en 2011 et développé par Edmund McMillen. C’est à ce jour un des roguelike les plus connus, puisqu'il est toujours mis à jour plus de 10 ans après. Ce jeu ainsi qu’Enter the gungeon, dont nous parlerons plus tard sont des twin stick shooter, ce qui signifie que le joueur contrôle le mouvement du personnage d’une main et avec l'autre, la direction vers laquelle les projectiles. L’origine du jeu est une { Game Jam sous flash, pendant laquelle le créateur} voulait développer un jeu qui n’attirerait personne. Une mécanique principale du jeu est les combinaisons d’items, qui sont à ce jour au nombre de 1200. Cela permet une quasi-infinité de combinaisons et une re-jouabilité poussée à l’extrême. La communauté alimente aussi le jeu de contenus, allant du simple mod pour avoir des descriptions des items, à des histoires complètes. Certaines de ces histoires ont même réussi à se faire implémenter dans des extensions du jeu.
\\
\vspace*{.1cm}
\normalsize{\subsubsection{Hadès}}
Hadès est un RPG roguelite développé et édité par Supergiant Games sorti en 2020. Il se présente en 3D isométrique. Comme Rogue, ce jeu peut également être qualifié de die and retry. Lors de la mort du personnage, le joueur perd tout l'équipement récupéré durant la partie à l'exception d'une monnaie d'échange permettant de débloquer des armes ou améliorations permanentes. Un seul personnage est jouable. Cependant, Hadès propose une option de personnalisation du personnage à travers le choix de son arme principale parmi les six existantes. L’histoire et les dialogues sont également mis en avant. 
\\
\newpage
\vspace*{1mm}
\normalsize{\subsubsection{Dead cells}}
Dead cells, sorti en 2018 et développé par Motion-Twin, est également un roguelike ainsi qu'un platformer. Cela signifie que pour parcourir les niveaux, il est nécessaire d'avoir recours à des sauts ou de l'escalade. Contrairement aux précédents exemples que nous présentons, Dead cells se présente en 2D avec des déplacements sur les axes X et Y. Le gameplay est rapide et exigeant, entraînant une grande difficulté et nécessitant une grande réactivité ainsi que de la concentration afin de ne pas mourir. Malgré la perte de l'équipement à chaque fin de partie, Dead cells propose, de manière similaire à Hadès, des améliorations permanentes afin de faciliter le commencement des prochaines parties. 
\\

\vspace*{.2mm}
\normalsize{\subsubsection{Nuclear Throne}}
Nuclear Throne est un roguelike paru en 2015 et développé par Vlambeer. Il est assez similaire à The Binding of Isaac. Contrairement à ce dernier, la réussite ou l'échec d'une partie ne repose que très rarement sur la chance, mais davantage sur les compétences du joueur. Ce jeu, comme Dead cells, est exigeant et punitif. Les loots aléatoires et la génération procédurale ne sont pas les causes de défaites. Celles-ci ne sont causées que par une mauvaise action du joueur. Les parties de Nuclear Throne sont généralement plus courtes que celles de The Binding of Isaac, mais requièrent tout autant de la concentration. Malgré la génération procédurale des niveaux, l'aléatoire est contrôlé afin de proposer des niveaux équilibrés. 
\\
\vspace*{.2mm}
\normalsize{\subsubsection{Slay the spire }}
Slay the spire s'inscrit également dans le genre roguelike. Ce jeu a été publié en 2019 par MegaCrit. Il se caractérise par l'absence d'une carte où le joueur pourrait se déplacer. Il s'agit simplement d'une succession de salles contenant des ennemis ou proposant des améliorations.Les affrontements se déroulent au tour par tour et utilisent une mécanique de deck-building. Au début du jeu, le joueur reçoit un premier paquet de 10 cartes en fonction du personnage qu'il a choisi. Il peut ensuite, au cours de sa partie, récupérer de nouvelles cartes sur des ennemis puissants, en acheter, améliorer ses cartes actuelles ou se débarrasser de certaines afin de les remplacer par des meilleures. 
\\
\newpage
\vspace*{.2mm}
\normalsize{\subsubsection{Enter the gungeon}}
Enter the gungeon est un roguelike qui rentre aussi dans le genre du bullet hell. Ce dernier est un genre de jeu qui se caractérise par une grande quantité de projectiles présents à l’écran que le joueur doit éviter. Ce jeu a été développé par Dodge Roll en 2016. Enter the gungeon est radicalement différent dans sa manière d’être joué de The Binding of Isaac notamment à cause de l’aspect très rapide et réactif du jeu. Une mécanique importante du jeu et qui améliore son rythme est l’esquive (en anglais dodge roll, comme le développeur), qui est primordial à la réussite du jeu, permettant de sauter par-dessus des obstacles ou des projectiles inévitables autrement. Beaucoup de salles sont équipées de téléporteurs qui permettent une navigation plus rapide des niveaux. Ces derniers sont plus longs que les étages de The Binding of Isaac. Par exemple, il faut environ 10 à 15 minutes pour finir un niveau d'Enter the gungeon contre 5 à 7 minutes environ  pour chaque étage de The Binding of Isaac. 
\\
\vspace*{1mm}
\section{Découpage du projet}
\subsection{\textbf {Répartition des tâches}}
\vspace{1cm}
%------------Tableau-----------------------
\begin{tabular}{|c|c|c|c|c|}
\hline
                      & Thomas      & Arnaud      & Axelle      & Gwennan     \\ \hline
Graphisme             & Suppléant   &             &             & Responsable \\ \hline
Immersion             & Responsable &             & Suppléante  &             \\ \hline
Level Design          & Responsable & Suppléant   &             &             \\ \hline
IA                    &             & Responsable & Suppléante  &             \\ \hline
Gameplay              &             &             & Responsable & Suppléante  \\ \hline
Gestion des Contrôles &             &             & Responsable & Suppléante  \\ \hline
Réseau                & Suppléant   & Responsable &             &             \\ \hline
Communication         &             & Suppléant   &             & Responsable \\ \hline
\end{tabular}
\\
\newpage
\vspace*{1mm}
\subsection{\textbf {Avancement du projet}}
\vspace*{.2cm}
\normalsize{\subsubsection{Première soutenance}}
\vspace*{1cm}
\begin{tabular}{|c|c|c|}
\hline
                              & Details                              & 1ère soutenance \\ \hline
\multirow{3}{*}{Graphisme}    & Création  personnages, ennemis ect.. & 25\%            \\ \cline{2-3} 
                              & Animation des personnages            & 30\%            \\ \cline{2-3} 
                              & Création des matériaux/textures      & 50\%            \\ \hline
\multirow{2}{*}{Immersion}    & HUD                                  & 25\%            \\ \cline{2-3} 
                              & Histoire                             & 80\%            \\ \hline
\multirow{3}{*}{Level Design} & Génération procédurale               & 80\%            \\ \cline{2-3} 
                              & Création des niveaux de difficultés  & 20\%            \\ \cline{2-3} 
                              & Création de layout                   & 20\%            \\ \hline
IA                            & Comportement des ennemis             & 10\%            \\ \hline
\multirow{3}{*}{Gameplay}     & Création des statistiques            & 20\%            \\ \cline{2-3} 
                              & Déplacements                         & 80\%            \\ \cline{2-3} 
                              & Attaques                             & 30\%            \\ \hline
Gestion des contrôles         & Support qwerty et azerty             & 50\%            \\ \hline
Réseau                        & Multijoueur                          & 0\%             \\ \hline
\end{tabular}
\\
\vspace*{\fill}

\\
\vspace*{1mm}
\normalsize{\subsubsection{Deuxième soutenance}}
\vspace{1cm}

\begin{tabular}{|c|c|c|}
\hline
                              & Details                              & 2ème soutenance \\ \hline
\multirow{3}{*}{Graphisme}    & Création  personnages, ennemis ect.. & 70\%            \\ \cline{2-3} 
                              & Animation des personnages            & 70\%            \\ \cline{2-3} 
                              & Création des matériaux/textures      & 75\%            \\ \hline
\multirow{2}{*}{Immersion}    & HUD                                  & 75\%            \\ \cline{2-3} 
                              & Histoire                             & 100\%           \\ \hline
\multirow{3}{*}{Level Design} & Génération procédurale               & 100\%           \\ \cline{2-3} 
                              & Création des niveaux de difficultés  & 50\%            \\ \cline{2-3} 
                              & Création de layout                   & 60\%            \\ \hline
IA                            & Comportement des ennemis             & 60\%            \\ \hline
\multirow{3}{*}{Gameplay}     & Création des statistiques            & 70\%            \\ \cline{2-3} 
                              & Déplacements                         & 90\%            \\ \cline{2-3} 
                              & Attaques                             & 70\%            \\ \hline
Gestion des contrôles         & Support qwerty et azerty             & 50\%            \\ \hline
Réseau                        & Multijoueur                          & 20\%            \\ \hline
\end{tabular}
\vspace*{\fill}
\\
\newpage
\vspace*{1mm}
\normalsize{\subsubsection{Troisième et dernière soutenance}}
\vspace{1cm}
\begin{tabular}{|c|c|c|}
\hline
                              & Details                              & 3ème soutenance \\ \hline
\multirow{3}{*}{Graphisme}    & Création  personnages, ennemis ect.. & 100\%           \\ \cline{2-3} 
                              & Animation des personnages            & 100\%           \\ \cline{2-3} 
                              & Création des matériaux/textures      & 100\%           \\ \hline
\multirow{2}{*}{Immersion}    & HUD                                  & 100\%           \\ \cline{2-3} 
                              & Histoire                             & 100\%           \\ \hline
\multirow{3}{*}{Level Design} & Génération procédurale               & 100\%           \\ \cline{2-3} 
                              & Création des niveaux de difficultés  & 100\%           \\ \cline{2-3} 
                              & Création de layout                   & 100\%           \\ \hline
IA                            & Comportement des ennemis             & 100\%           \\ \hline
\multirow{3}{*}{Gameplay}     & Création des statistiques            & 100\%           \\ \cline{2-3} 
                              & Déplacements                         & 100\%           \\ \cline{2-3} 
                              & Attaques                             & 100\%           \\ \hline
Gestion des contrôles         & Support qwerty et azerty             & 100\%           \\ \hline
Réseau                        & Multijoueur                          & 100\%           \\ \hline
\end{tabular}
\newpage
\vspace*{1mm} 
\section{Fonctionnel}

\paragraph{}
Pour mener à bien ce projet, différents outils nous seront nécessaires. 
\\
Tout d’abord, Discord sera l’outil privilégié pour communiquer entre les membres de la TAAG TEAM. 
\\
Nous utiliserons Rider et Visual Studio Code afin de coder le jeu et Unity pour le mettre en forme. 
\\
Des logiciels graphiques comme Photoshop et autres seront nécessaires afin de réaliser le design complet du jeu. 
\\
GitHub nous servira de Git repository et Git Kraken nous permettra une plus grande lisibilité de l’arborescence du projet. 
\\
Enfin, Overleaf nous permettra, comme ici, d’assurer une présentation propre et claire pour rédiger en Latex nos documents. 

\vspace*{.2cm}
\section{Opérationnel}
\paragraph{}
Du point de vue opérationnel, les coûts liés à la réalisation et au fonctionnement du jeu comme l’accès aux logiciels payants et l’hébergement du jeu et de son site web devraient être assurés par les bénéfices récoltés. En effet, afin de rendre le jeu lucratif, nous voulons avant le lancement, faire appel à du crowdfunding puis, dès sa sortie, le rendre payant. Les joueurs verseront donc dès le départ une certaine somme d’argent inférieure à dix euros. De plus, les éléments bonus comme des skins plus variés seront, eux aussi, disponibles en payant. L’argent perçu devrait ainsi suffire à combler les frais investis dans le jeu et même à financer ses développeurs.
%-----------Fin--------------
\newpage
    \vspace*{\fill}
    \hspace*{\fill} 
    \\
    \includegraphics[width = 15cm]{Image/final-logo.png}
    \\
    \hspace*{\fill}
    \vspace*{\fill}
     \\
\end{document}
